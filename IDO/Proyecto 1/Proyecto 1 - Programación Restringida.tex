\documentclass[11pt]{article}

    \usepackage[breakable]{tcolorbox}
    \usepackage{parskip} % Stop auto-indenting (to mimic markdown behaviour)
    
    \usepackage{iftex}
    \ifPDFTeX
    	\usepackage[T1]{fontenc}
    	\usepackage{mathpazo}
    \else
    	\usepackage{fontspec}
    \fi

    % Basic figure setup, for now with no caption control since it's done
    % automatically by Pandoc (which extracts ![](path) syntax from Markdown).
    \usepackage{graphicx}
    % Maintain compatibility with old templates. Remove in nbconvert 6.0
    \let\Oldincludegraphics\includegraphics
    % Ensure that by default, figures have no caption (until we provide a
    % proper Figure object with a Caption API and a way to capture that
    % in the conversion process - todo).
    \usepackage{caption}
    \DeclareCaptionFormat{nocaption}{}
    \captionsetup{format=nocaption,aboveskip=0pt,belowskip=0pt}

    \usepackage[Export]{adjustbox} % Used to constrain images to a maximum size
    \adjustboxset{max size={0.9\linewidth}{0.9\paperheight}}
    \usepackage{float}
    \floatplacement{figure}{H} % forces figures to be placed at the correct location
    \usepackage{xcolor} % Allow colors to be defined
    \usepackage{enumerate} % Needed for markdown enumerations to work
    \usepackage{geometry} % Used to adjust the document margins
    \usepackage{amsmath} % Equations
    \usepackage{amssymb} % Equations
    \usepackage{textcomp} % defines textquotesingle
    % Hack from http://tex.stackexchange.com/a/47451/13684:
    \AtBeginDocument{%
        \def\PYZsq{\textquotesingle}% Upright quotes in Pygmentized code
    }
    \usepackage{upquote} % Upright quotes for verbatim code
    \usepackage{eurosym} % defines \euro
    \usepackage[mathletters]{ucs} % Extended unicode (utf-8) support
    \usepackage{fancyvrb} % verbatim replacement that allows latex
    \usepackage{grffile} % extends the file name processing of package graphics 
                         % to support a larger range
    \makeatletter % fix for grffile with XeLaTeX
    \def\Gread@@xetex#1{%
      \IfFileExists{"\Gin@base".bb}%
      {\Gread@eps{\Gin@base.bb}}%
      {\Gread@@xetex@aux#1}%
    }
    \makeatother

    % The hyperref package gives us a pdf with properly built
    % internal navigation ('pdf bookmarks' for the table of contents,
    % internal cross-reference links, web links for URLs, etc.)
    \usepackage{hyperref}
    % The default LaTeX title has an obnoxious amount of whitespace. By default,
    % titling removes some of it. It also provides customization options.
    \usepackage{titling}
    \usepackage{longtable} % longtable support required by pandoc >1.10
    \usepackage{booktabs}  % table support for pandoc > 1.12.2
    \usepackage[inline]{enumitem} % IRkernel/repr support (it uses the enumerate* environment)
    \usepackage[normalem]{ulem} % ulem is needed to support strikethroughs (\sout)
                                % normalem makes italics be italics, not underlines
    \usepackage{mathrsfs}
    

    
    % Colors for the hyperref package
    \definecolor{urlcolor}{rgb}{0,.145,.698}
    \definecolor{linkcolor}{rgb}{.71,0.21,0.01}
    \definecolor{citecolor}{rgb}{.12,.54,.11}

    % ANSI colors
    \definecolor{ansi-black}{HTML}{3E424D}
    \definecolor{ansi-black-intense}{HTML}{282C36}
    \definecolor{ansi-red}{HTML}{E75C58}
    \definecolor{ansi-red-intense}{HTML}{B22B31}
    \definecolor{ansi-green}{HTML}{00A250}
    \definecolor{ansi-green-intense}{HTML}{007427}
    \definecolor{ansi-yellow}{HTML}{DDB62B}
    \definecolor{ansi-yellow-intense}{HTML}{B27D12}
    \definecolor{ansi-blue}{HTML}{208FFB}
    \definecolor{ansi-blue-intense}{HTML}{0065CA}
    \definecolor{ansi-magenta}{HTML}{D160C4}
    \definecolor{ansi-magenta-intense}{HTML}{A03196}
    \definecolor{ansi-cyan}{HTML}{60C6C8}
    \definecolor{ansi-cyan-intense}{HTML}{258F8F}
    \definecolor{ansi-white}{HTML}{C5C1B4}
    \definecolor{ansi-white-intense}{HTML}{A1A6B2}
    \definecolor{ansi-default-inverse-fg}{HTML}{FFFFFF}
    \definecolor{ansi-default-inverse-bg}{HTML}{000000}

    % commands and environments needed by pandoc snippets
    % extracted from the output of `pandoc -s`
    \providecommand{\tightlist}{%
      \setlength{\itemsep}{0pt}\setlength{\parskip}{0pt}}
    \DefineVerbatimEnvironment{Highlighting}{Verbatim}{commandchars=\\\{\}}
    % Add ',fontsize=\small' for more characters per line
    \newenvironment{Shaded}{}{}
    \newcommand{\KeywordTok}[1]{\textcolor[rgb]{0.00,0.44,0.13}{\textbf{{#1}}}}
    \newcommand{\DataTypeTok}[1]{\textcolor[rgb]{0.56,0.13,0.00}{{#1}}}
    \newcommand{\DecValTok}[1]{\textcolor[rgb]{0.25,0.63,0.44}{{#1}}}
    \newcommand{\BaseNTok}[1]{\textcolor[rgb]{0.25,0.63,0.44}{{#1}}}
    \newcommand{\FloatTok}[1]{\textcolor[rgb]{0.25,0.63,0.44}{{#1}}}
    \newcommand{\CharTok}[1]{\textcolor[rgb]{0.25,0.44,0.63}{{#1}}}
    \newcommand{\StringTok}[1]{\textcolor[rgb]{0.25,0.44,0.63}{{#1}}}
    \newcommand{\CommentTok}[1]{\textcolor[rgb]{0.38,0.63,0.69}{\textit{{#1}}}}
    \newcommand{\OtherTok}[1]{\textcolor[rgb]{0.00,0.44,0.13}{{#1}}}
    \newcommand{\AlertTok}[1]{\textcolor[rgb]{1.00,0.00,0.00}{\textbf{{#1}}}}
    \newcommand{\FunctionTok}[1]{\textcolor[rgb]{0.02,0.16,0.49}{{#1}}}
    \newcommand{\RegionMarkerTok}[1]{{#1}}
    \newcommand{\ErrorTok}[1]{\textcolor[rgb]{1.00,0.00,0.00}{\textbf{{#1}}}}
    \newcommand{\NormalTok}[1]{{#1}}
    
    % Additional commands for more recent versions of Pandoc
    \newcommand{\ConstantTok}[1]{\textcolor[rgb]{0.53,0.00,0.00}{{#1}}}
    \newcommand{\SpecialCharTok}[1]{\textcolor[rgb]{0.25,0.44,0.63}{{#1}}}
    \newcommand{\VerbatimStringTok}[1]{\textcolor[rgb]{0.25,0.44,0.63}{{#1}}}
    \newcommand{\SpecialStringTok}[1]{\textcolor[rgb]{0.73,0.40,0.53}{{#1}}}
    \newcommand{\ImportTok}[1]{{#1}}
    \newcommand{\DocumentationTok}[1]{\textcolor[rgb]{0.73,0.13,0.13}{\textit{{#1}}}}
    \newcommand{\AnnotationTok}[1]{\textcolor[rgb]{0.38,0.63,0.69}{\textbf{\textit{{#1}}}}}
    \newcommand{\CommentVarTok}[1]{\textcolor[rgb]{0.38,0.63,0.69}{\textbf{\textit{{#1}}}}}
    \newcommand{\VariableTok}[1]{\textcolor[rgb]{0.10,0.09,0.49}{{#1}}}
    \newcommand{\ControlFlowTok}[1]{\textcolor[rgb]{0.00,0.44,0.13}{\textbf{{#1}}}}
    \newcommand{\OperatorTok}[1]{\textcolor[rgb]{0.40,0.40,0.40}{{#1}}}
    \newcommand{\BuiltInTok}[1]{{#1}}
    \newcommand{\ExtensionTok}[1]{{#1}}
    \newcommand{\PreprocessorTok}[1]{\textcolor[rgb]{0.74,0.48,0.00}{{#1}}}
    \newcommand{\AttributeTok}[1]{\textcolor[rgb]{0.49,0.56,0.16}{{#1}}}
    \newcommand{\InformationTok}[1]{\textcolor[rgb]{0.38,0.63,0.69}{\textbf{\textit{{#1}}}}}
    \newcommand{\WarningTok}[1]{\textcolor[rgb]{0.38,0.63,0.69}{\textbf{\textit{{#1}}}}}
    
    
    % Define a nice break command that doesn't care if a line doesn't already
    % exist.
    \def\br{\hspace*{\fill} \\* }
    % Math Jax compatibility definitions
    \def\gt{>}
    \def\lt{<}
    \let\Oldtex\TeX
    \let\Oldlatex\LaTeX
    \renewcommand{\TeX}{\textrm{\Oldtex}}
    \renewcommand{\LaTeX}{\textrm{\Oldlatex}}
    % Document parameters
    % Document title
    \title{Proyecto 1: Programación Restringida}
    \author{Mauricio Trejo \and Andrés Ángeles}
    \date{Octubre 2019}
    
% Pygments definitions
\makeatletter
\def\PY@reset{\let\PY@it=\relax \let\PY@bf=\relax%
    \let\PY@ul=\relax \let\PY@tc=\relax%
    \let\PY@bc=\relax \let\PY@ff=\relax}
\def\PY@tok#1{\csname PY@tok@#1\endcsname}
\def\PY@toks#1+{\ifx\relax#1\empty\else%
    \PY@tok{#1}\expandafter\PY@toks\fi}
\def\PY@do#1{\PY@bc{\PY@tc{\PY@ul{%
    \PY@it{\PY@bf{\PY@ff{#1}}}}}}}
\def\PY#1#2{\PY@reset\PY@toks#1+\relax+\PY@do{#2}}

\expandafter\def\csname PY@tok@w\endcsname{\def\PY@tc##1{\textcolor[rgb]{0.73,0.73,0.73}{##1}}}
\expandafter\def\csname PY@tok@c\endcsname{\let\PY@it=\textit\def\PY@tc##1{\textcolor[rgb]{0.25,0.50,0.50}{##1}}}
\expandafter\def\csname PY@tok@cp\endcsname{\def\PY@tc##1{\textcolor[rgb]{0.74,0.48,0.00}{##1}}}
\expandafter\def\csname PY@tok@k\endcsname{\let\PY@bf=\textbf\def\PY@tc##1{\textcolor[rgb]{0.00,0.50,0.00}{##1}}}
\expandafter\def\csname PY@tok@kp\endcsname{\def\PY@tc##1{\textcolor[rgb]{0.00,0.50,0.00}{##1}}}
\expandafter\def\csname PY@tok@kt\endcsname{\def\PY@tc##1{\textcolor[rgb]{0.69,0.00,0.25}{##1}}}
\expandafter\def\csname PY@tok@o\endcsname{\def\PY@tc##1{\textcolor[rgb]{0.40,0.40,0.40}{##1}}}
\expandafter\def\csname PY@tok@ow\endcsname{\let\PY@bf=\textbf\def\PY@tc##1{\textcolor[rgb]{0.67,0.13,1.00}{##1}}}
\expandafter\def\csname PY@tok@nb\endcsname{\def\PY@tc##1{\textcolor[rgb]{0.00,0.50,0.00}{##1}}}
\expandafter\def\csname PY@tok@nf\endcsname{\def\PY@tc##1{\textcolor[rgb]{0.00,0.00,1.00}{##1}}}
\expandafter\def\csname PY@tok@nc\endcsname{\let\PY@bf=\textbf\def\PY@tc##1{\textcolor[rgb]{0.00,0.00,1.00}{##1}}}
\expandafter\def\csname PY@tok@nn\endcsname{\let\PY@bf=\textbf\def\PY@tc##1{\textcolor[rgb]{0.00,0.00,1.00}{##1}}}
\expandafter\def\csname PY@tok@ne\endcsname{\let\PY@bf=\textbf\def\PY@tc##1{\textcolor[rgb]{0.82,0.25,0.23}{##1}}}
\expandafter\def\csname PY@tok@nv\endcsname{\def\PY@tc##1{\textcolor[rgb]{0.10,0.09,0.49}{##1}}}
\expandafter\def\csname PY@tok@no\endcsname{\def\PY@tc##1{\textcolor[rgb]{0.53,0.00,0.00}{##1}}}
\expandafter\def\csname PY@tok@nl\endcsname{\def\PY@tc##1{\textcolor[rgb]{0.63,0.63,0.00}{##1}}}
\expandafter\def\csname PY@tok@ni\endcsname{\let\PY@bf=\textbf\def\PY@tc##1{\textcolor[rgb]{0.60,0.60,0.60}{##1}}}
\expandafter\def\csname PY@tok@na\endcsname{\def\PY@tc##1{\textcolor[rgb]{0.49,0.56,0.16}{##1}}}
\expandafter\def\csname PY@tok@nt\endcsname{\let\PY@bf=\textbf\def\PY@tc##1{\textcolor[rgb]{0.00,0.50,0.00}{##1}}}
\expandafter\def\csname PY@tok@nd\endcsname{\def\PY@tc##1{\textcolor[rgb]{0.67,0.13,1.00}{##1}}}
\expandafter\def\csname PY@tok@s\endcsname{\def\PY@tc##1{\textcolor[rgb]{0.73,0.13,0.13}{##1}}}
\expandafter\def\csname PY@tok@sd\endcsname{\let\PY@it=\textit\def\PY@tc##1{\textcolor[rgb]{0.73,0.13,0.13}{##1}}}
\expandafter\def\csname PY@tok@si\endcsname{\let\PY@bf=\textbf\def\PY@tc##1{\textcolor[rgb]{0.73,0.40,0.53}{##1}}}
\expandafter\def\csname PY@tok@se\endcsname{\let\PY@bf=\textbf\def\PY@tc##1{\textcolor[rgb]{0.73,0.40,0.13}{##1}}}
\expandafter\def\csname PY@tok@sr\endcsname{\def\PY@tc##1{\textcolor[rgb]{0.73,0.40,0.53}{##1}}}
\expandafter\def\csname PY@tok@ss\endcsname{\def\PY@tc##1{\textcolor[rgb]{0.10,0.09,0.49}{##1}}}
\expandafter\def\csname PY@tok@sx\endcsname{\def\PY@tc##1{\textcolor[rgb]{0.00,0.50,0.00}{##1}}}
\expandafter\def\csname PY@tok@m\endcsname{\def\PY@tc##1{\textcolor[rgb]{0.40,0.40,0.40}{##1}}}
\expandafter\def\csname PY@tok@gh\endcsname{\let\PY@bf=\textbf\def\PY@tc##1{\textcolor[rgb]{0.00,0.00,0.50}{##1}}}
\expandafter\def\csname PY@tok@gu\endcsname{\let\PY@bf=\textbf\def\PY@tc##1{\textcolor[rgb]{0.50,0.00,0.50}{##1}}}
\expandafter\def\csname PY@tok@gd\endcsname{\def\PY@tc##1{\textcolor[rgb]{0.63,0.00,0.00}{##1}}}
\expandafter\def\csname PY@tok@gi\endcsname{\def\PY@tc##1{\textcolor[rgb]{0.00,0.63,0.00}{##1}}}
\expandafter\def\csname PY@tok@gr\endcsname{\def\PY@tc##1{\textcolor[rgb]{1.00,0.00,0.00}{##1}}}
\expandafter\def\csname PY@tok@ge\endcsname{\let\PY@it=\textit}
\expandafter\def\csname PY@tok@gs\endcsname{\let\PY@bf=\textbf}
\expandafter\def\csname PY@tok@gp\endcsname{\let\PY@bf=\textbf\def\PY@tc##1{\textcolor[rgb]{0.00,0.00,0.50}{##1}}}
\expandafter\def\csname PY@tok@go\endcsname{\def\PY@tc##1{\textcolor[rgb]{0.53,0.53,0.53}{##1}}}
\expandafter\def\csname PY@tok@gt\endcsname{\def\PY@tc##1{\textcolor[rgb]{0.00,0.27,0.87}{##1}}}
\expandafter\def\csname PY@tok@err\endcsname{\def\PY@bc##1{\setlength{\fboxsep}{0pt}\fcolorbox[rgb]{1.00,0.00,0.00}{1,1,1}{\strut ##1}}}
\expandafter\def\csname PY@tok@kc\endcsname{\let\PY@bf=\textbf\def\PY@tc##1{\textcolor[rgb]{0.00,0.50,0.00}{##1}}}
\expandafter\def\csname PY@tok@kd\endcsname{\let\PY@bf=\textbf\def\PY@tc##1{\textcolor[rgb]{0.00,0.50,0.00}{##1}}}
\expandafter\def\csname PY@tok@kn\endcsname{\let\PY@bf=\textbf\def\PY@tc##1{\textcolor[rgb]{0.00,0.50,0.00}{##1}}}
\expandafter\def\csname PY@tok@kr\endcsname{\let\PY@bf=\textbf\def\PY@tc##1{\textcolor[rgb]{0.00,0.50,0.00}{##1}}}
\expandafter\def\csname PY@tok@bp\endcsname{\def\PY@tc##1{\textcolor[rgb]{0.00,0.50,0.00}{##1}}}
\expandafter\def\csname PY@tok@fm\endcsname{\def\PY@tc##1{\textcolor[rgb]{0.00,0.00,1.00}{##1}}}
\expandafter\def\csname PY@tok@vc\endcsname{\def\PY@tc##1{\textcolor[rgb]{0.10,0.09,0.49}{##1}}}
\expandafter\def\csname PY@tok@vg\endcsname{\def\PY@tc##1{\textcolor[rgb]{0.10,0.09,0.49}{##1}}}
\expandafter\def\csname PY@tok@vi\endcsname{\def\PY@tc##1{\textcolor[rgb]{0.10,0.09,0.49}{##1}}}
\expandafter\def\csname PY@tok@vm\endcsname{\def\PY@tc##1{\textcolor[rgb]{0.10,0.09,0.49}{##1}}}
\expandafter\def\csname PY@tok@sa\endcsname{\def\PY@tc##1{\textcolor[rgb]{0.73,0.13,0.13}{##1}}}
\expandafter\def\csname PY@tok@sb\endcsname{\def\PY@tc##1{\textcolor[rgb]{0.73,0.13,0.13}{##1}}}
\expandafter\def\csname PY@tok@sc\endcsname{\def\PY@tc##1{\textcolor[rgb]{0.73,0.13,0.13}{##1}}}
\expandafter\def\csname PY@tok@dl\endcsname{\def\PY@tc##1{\textcolor[rgb]{0.73,0.13,0.13}{##1}}}
\expandafter\def\csname PY@tok@s2\endcsname{\def\PY@tc##1{\textcolor[rgb]{0.73,0.13,0.13}{##1}}}
\expandafter\def\csname PY@tok@sh\endcsname{\def\PY@tc##1{\textcolor[rgb]{0.73,0.13,0.13}{##1}}}
\expandafter\def\csname PY@tok@s1\endcsname{\def\PY@tc##1{\textcolor[rgb]{0.73,0.13,0.13}{##1}}}
\expandafter\def\csname PY@tok@mb\endcsname{\def\PY@tc##1{\textcolor[rgb]{0.40,0.40,0.40}{##1}}}
\expandafter\def\csname PY@tok@mf\endcsname{\def\PY@tc##1{\textcolor[rgb]{0.40,0.40,0.40}{##1}}}
\expandafter\def\csname PY@tok@mh\endcsname{\def\PY@tc##1{\textcolor[rgb]{0.40,0.40,0.40}{##1}}}
\expandafter\def\csname PY@tok@mi\endcsname{\def\PY@tc##1{\textcolor[rgb]{0.40,0.40,0.40}{##1}}}
\expandafter\def\csname PY@tok@il\endcsname{\def\PY@tc##1{\textcolor[rgb]{0.40,0.40,0.40}{##1}}}
\expandafter\def\csname PY@tok@mo\endcsname{\def\PY@tc##1{\textcolor[rgb]{0.40,0.40,0.40}{##1}}}
\expandafter\def\csname PY@tok@ch\endcsname{\let\PY@it=\textit\def\PY@tc##1{\textcolor[rgb]{0.25,0.50,0.50}{##1}}}
\expandafter\def\csname PY@tok@cm\endcsname{\let\PY@it=\textit\def\PY@tc##1{\textcolor[rgb]{0.25,0.50,0.50}{##1}}}
\expandafter\def\csname PY@tok@cpf\endcsname{\let\PY@it=\textit\def\PY@tc##1{\textcolor[rgb]{0.25,0.50,0.50}{##1}}}
\expandafter\def\csname PY@tok@c1\endcsname{\let\PY@it=\textit\def\PY@tc##1{\textcolor[rgb]{0.25,0.50,0.50}{##1}}}
\expandafter\def\csname PY@tok@cs\endcsname{\let\PY@it=\textit\def\PY@tc##1{\textcolor[rgb]{0.25,0.50,0.50}{##1}}}

\def\PYZbs{\char`\\}
\def\PYZus{\char`\_}
\def\PYZob{\char`\{}
\def\PYZcb{\char`\}}
\def\PYZca{\char`\^}
\def\PYZam{\char`\&}
\def\PYZlt{\char`\<}
\def\PYZgt{\char`\>}
\def\PYZsh{\char`\#}
\def\PYZpc{\char`\%}
\def\PYZdl{\char`\$}
\def\PYZhy{\char`\-}
\def\PYZsq{\char`\'}
\def\PYZdq{\char`\"}
\def\PYZti{\char`\~}
% for compatibility with earlier versions
\def\PYZat{@}
\def\PYZlb{[}
\def\PYZrb{]}
\makeatother


    % For linebreaks inside Verbatim environment from package fancyvrb. 
    \makeatletter
        \newbox\Wrappedcontinuationbox 
        \newbox\Wrappedvisiblespacebox 
        \newcommand*\Wrappedvisiblespace {\textcolor{red}{\textvisiblespace}} 
        \newcommand*\Wrappedcontinuationsymbol {\textcolor{red}{\llap{\tiny$\m@th\hookrightarrow$}}} 
        \newcommand*\Wrappedcontinuationindent {3ex } 
        \newcommand*\Wrappedafterbreak {\kern\Wrappedcontinuationindent\copy\Wrappedcontinuationbox} 
        % Take advantage of the already applied Pygments mark-up to insert 
        % potential linebreaks for TeX processing. 
        %        {, <, #, %, $, ' and ": go to next line. 
        %        _, }, ^, &, >, - and ~: stay at end of broken line. 
        % Use of \textquotesingle for straight quote. 
        \newcommand*\Wrappedbreaksatspecials {% 
            \def\PYGZus{\discretionary{\char`\_}{\Wrappedafterbreak}{\char`\_}}% 
            \def\PYGZob{\discretionary{}{\Wrappedafterbreak\char`\{}{\char`\{}}% 
            \def\PYGZcb{\discretionary{\char`\}}{\Wrappedafterbreak}{\char`\}}}% 
            \def\PYGZca{\discretionary{\char`\^}{\Wrappedafterbreak}{\char`\^}}% 
            \def\PYGZam{\discretionary{\char`\&}{\Wrappedafterbreak}{\char`\&}}% 
            \def\PYGZlt{\discretionary{}{\Wrappedafterbreak\char`\<}{\char`\<}}% 
            \def\PYGZgt{\discretionary{\char`\>}{\Wrappedafterbreak}{\char`\>}}% 
            \def\PYGZsh{\discretionary{}{\Wrappedafterbreak\char`\#}{\char`\#}}% 
            \def\PYGZpc{\discretionary{}{\Wrappedafterbreak\char`\%}{\char`\%}}% 
            \def\PYGZdl{\discretionary{}{\Wrappedafterbreak\char`\$}{\char`\$}}% 
            \def\PYGZhy{\discretionary{\char`\-}{\Wrappedafterbreak}{\char`\-}}% 
            \def\PYGZsq{\discretionary{}{\Wrappedafterbreak\textquotesingle}{\textquotesingle}}% 
            \def\PYGZdq{\discretionary{}{\Wrappedafterbreak\char`\"}{\char`\"}}% 
            \def\PYGZti{\discretionary{\char`\~}{\Wrappedafterbreak}{\char`\~}}% 
        } 
        % Some characters . , ; ? ! / are not pygmentized. 
        % This macro makes them "active" and they will insert potential linebreaks 
        \newcommand*\Wrappedbreaksatpunct {% 
            \lccode`\~`\.\lowercase{\def~}{\discretionary{\hbox{\char`\.}}{\Wrappedafterbreak}{\hbox{\char`\.}}}% 
            \lccode`\~`\,\lowercase{\def~}{\discretionary{\hbox{\char`\,}}{\Wrappedafterbreak}{\hbox{\char`\,}}}% 
            \lccode`\~`\;\lowercase{\def~}{\discretionary{\hbox{\char`\;}}{\Wrappedafterbreak}{\hbox{\char`\;}}}% 
            \lccode`\~`\:\lowercase{\def~}{\discretionary{\hbox{\char`\:}}{\Wrappedafterbreak}{\hbox{\char`\:}}}% 
            \lccode`\~`\?\lowercase{\def~}{\discretionary{\hbox{\char`\?}}{\Wrappedafterbreak}{\hbox{\char`\?}}}% 
            \lccode`\~`\!\lowercase{\def~}{\discretionary{\hbox{\char`\!}}{\Wrappedafterbreak}{\hbox{\char`\!}}}% 
            \lccode`\~`\/\lowercase{\def~}{\discretionary{\hbox{\char`\/}}{\Wrappedafterbreak}{\hbox{\char`\/}}}% 
            \catcode`\.\active
            \catcode`\,\active 
            \catcode`\;\active
            \catcode`\:\active
            \catcode`\?\active
            \catcode`\!\active
            \catcode`\/\active 
            \lccode`\~`\~ 	
        }
    \makeatother

    \let\OriginalVerbatim=\Verbatim
    \makeatletter
    \renewcommand{\Verbatim}[1][1]{%
        %\parskip\z@skip
        \sbox\Wrappedcontinuationbox {\Wrappedcontinuationsymbol}%
        \sbox\Wrappedvisiblespacebox {\FV@SetupFont\Wrappedvisiblespace}%
        \def\FancyVerbFormatLine ##1{\hsize\linewidth
            \vtop{\raggedright\hyphenpenalty\z@\exhyphenpenalty\z@
                \doublehyphendemerits\z@\finalhyphendemerits\z@
                \strut ##1\strut}%
        }%
        % If the linebreak is at a space, the latter will be displayed as visible
        % space at end of first line, and a continuation symbol starts next line.
        % Stretch/shrink are however usually zero for typewriter font.
        \def\FV@Space {%
            \nobreak\hskip\z@ plus\fontdimen3\font minus\fontdimen4\font
            \discretionary{\copy\Wrappedvisiblespacebox}{\Wrappedafterbreak}
            {\kern\fontdimen2\font}%
        }%
        
        % Allow breaks at special characters using \PYG... macros.
        \Wrappedbreaksatspecials
        % Breaks at punctuation characters . , ; ? ! and / need catcode=\active 	
        \OriginalVerbatim[#1,codes*=\Wrappedbreaksatpunct]%
    }
    \makeatother

    % Exact colors from NB
    \definecolor{incolor}{HTML}{303F9F}
    \definecolor{outcolor}{HTML}{D84315}
    \definecolor{cellborder}{HTML}{CFCFCF}
    \definecolor{cellbackground}{HTML}{F7F7F7}
    
    % prompt
    \makeatletter
    \newcommand{\boxspacing}{\kern\kvtcb@left@rule\kern\kvtcb@boxsep}
    \makeatother
    \newcommand{\prompt}[4]{
        \ttfamily\llap{{\color{#2}[#3]:\hspace{3pt}#4}}\vspace{-\baselineskip}
    }
    
    % Prevent overflowing lines due to hard-to-break entities
    \sloppy 
    % Setup hyperref package
    \hypersetup{
      breaklinks=true,  % so long urls are correctly broken across lines
      colorlinks=true,
      urlcolor=urlcolor,
      linkcolor=linkcolor,
      citecolor=citecolor,
      }
    % Slightly bigger margins than the latex defaults
    
    \geometry{verbose,tmargin=1in,bmargin=1in,lmargin=1in,rmargin=1in}
    
    

\begin{document}
    
    \maketitle  
    
    \hypertarget{planteamiento-del-problema}{%
\section{Planteamiento del problema}\label{planteamiento-del-problema}}

    El dueño de una fabrica de dulces ha detectado caramelo sobrante durante
el proceso de fabricación, existen diferentes tipos de caramelo en la
cantidad de sobrante. El objetivo del empresario es obtener más
ganancias al fabricar un tipo de caramelo de calidad de exportación con
el sobrante obtenido y asi garantizar que su producto llegue al mercado.

Si arma paquetes que pesen a lo más 3kg no pagara ningún tipo de
arancel, el empaque tiene una capacidad de 1 \(dm^3\).

Para que el negocio sea redondo, la producción del paquete debe de
costar a lo más \$200 y además garantizar que el caramelo sea lo más
dulce posible. El objetivo es indicar el número óptimo de cada tipo de
caramelo en cada paquete a partir de la siguiente información:

    \begin{longtable}[]{@{}lllll@{}}
\toprule
Caramelo & Peso (gramos) & Dimensiones (cm) & Dulzura & Valor
(\$)\tabularnewline
\midrule
\endhead
A & 100 & 8 x 2.5 x 0.5 & 20 & 8\tabularnewline
B & 45 & 7 x 2 x 0.5 & 16 & 6.8\tabularnewline
C & 10 & 3 x 2 x 0.5 & 9 & 4\tabularnewline
D & 25 & 3 x 3 x 0.5 & 7 & 3\tabularnewline
\bottomrule
\end{longtable}

    \hypertarget{modelado-y-soluciuxf3n-del-problema-mediante-el-uso-de-programaciuxf3n-de-restricciones}{%
\section{Modelado y solución del problema mediante el uso de
Programación de
restricciones}\label{modelado-y-soluciuxf3n-del-problema-mediante-el-uso-de-programaciuxf3n-de-restricciones}}

    Definimos \(x_i\) como el número de caramelos del tipo \(i\) que
utilizamos en cada empaque donde \(i = 1, 2, 3, 4\).

El objetivo es que el caramelo del empaque sea lo más dulce posible, por
lo que la función objetivo queda dada por
\begin{equation}
    z = f(\bar{x}) = 20 x_1 + 16 x_2 + 9 x_3 + 7 x_4
\end{equation}
Además, sabemos que cada decímetro cúbico equivale a mil centímetros
cúbicos, por lo que la restricción del volumen del empaque es
\begin{equation}
    (8 \times 2.5 \times 0.5) x_1 + (7 \times 2 \times 0.5) x_2 + (3 \times 2 \times 0.5) x_3 + (3 \times 3 \times 0.5) x_4 =\\10 x_1 + 7 x_2 + 3 x_3 + 3 x_4 \leq 1000
\end{equation}
y el modelo del problema es

\begin{equation} \label{modelo}
    \begin{aligned}
        &\text{maximizar} && z = 20 x_1 + 16 x_2 + 9 x_3 + 7 x_4\\
        &\text{sujeto a} && 8 x_1 + 6.8 x_2 + 4 x_3 + 3 x_4 \leq 200\\
        &   &&10 x_1 + 7 x_2 + 3 x_3 + 3 x_4 \leq 1000\\
        &   && 100 x_1 + 45 x_2 + 10 x_3 + 25 x_4 \leq 3000\\
        &   && x_1, x_2, x_3, x_4 \in \mathbb{Z}^+ \cup \{0\}
    \end{aligned}
\end{equation}

    \hypertarget{buxfasqueda-de-los-dominios-restringidos-de-cada-variable}{%
\subsection{Búsqueda de los dominios restringidos de cada
variable}\label{buxfasqueda-de-los-dominios-restringidos-de-cada-variable}}

    Consideremos las restricciones del modelo:
\begin{align}
    8 x_1 + 6.8 x_2 + 4 x_3 + 3 x_4 &\leq 200\\
    10 x_1 + 7 x_2 + 3 x_3 + 3 x_4 &\leq 1000\\
    100 x_1 + 45 x_2 + 10 x_3 + 25 x_4 &\leq 3000\\
    x_1, x_2, x_3, x_4 \in \mathbb{Z}^+ &\cup \{0\}
\end{align}
Si igualamos \(x_2, x_3\) y \(x_4\) a cero, tenemos que
\(8 x_1 \leq 200\), \(10x_1 \leq 1000\) y \(100 x_1 \leq 3000\), por lo
que concluimos que
\begin{equation}
    x_1 \in \{0, 1, 2, \dots, 25\}
\end{equation}
De la misma forma, deducimos que
\begin{align}
    x_2 &\in \{0, 1, 2, \dots, 29\}\\
    x_3 &\in \{0, 1, 2, \dots, 50\}\\
    x_4 &\in \{0, 1, 2, \dots, 66\}
\end{align}
y planteamos la primera versión de nuestro problema restringido:

    \begin{tcolorbox}[breakable, size=fbox, boxrule=1pt, pad at break*=1mm,colback=cellbackground, colframe=cellborder]
\prompt{In}{incolor}{21}{\boxspacing}
\begin{Verbatim}[commandchars=\\\{\}]
\PY{k+kn}{from} \PY{n+nn}{constraint} \PY{k}{import} \PY{o}{*}
\PY{k+kn}{import} \PY{n+nn}{time}

\PY{c+c1}{\PYZsh{} Creando el problema y el rango de las variables:}
\PY{n}{problem} \PY{o}{=} \PY{n}{Problem}\PY{p}{(}\PY{p}{)}
\PY{n}{problem}\PY{o}{.}\PY{n}{addVariable}\PY{p}{(}\PY{l+s+s2}{\PYZdq{}}\PY{l+s+s2}{x1}\PY{l+s+s2}{\PYZdq{}}\PY{p}{,} \PY{n+nb}{range}\PY{p}{(}\PY{l+m+mi}{26}\PY{p}{)}\PY{p}{)}
\PY{n}{problem}\PY{o}{.}\PY{n}{addVariable}\PY{p}{(}\PY{l+s+s2}{\PYZdq{}}\PY{l+s+s2}{x2}\PY{l+s+s2}{\PYZdq{}}\PY{p}{,} \PY{n+nb}{range}\PY{p}{(}\PY{l+m+mi}{30}\PY{p}{)}\PY{p}{)}
\PY{n}{problem}\PY{o}{.}\PY{n}{addVariable}\PY{p}{(}\PY{l+s+s2}{\PYZdq{}}\PY{l+s+s2}{x3}\PY{l+s+s2}{\PYZdq{}}\PY{p}{,} \PY{n+nb}{range}\PY{p}{(}\PY{l+m+mi}{51}\PY{p}{)}\PY{p}{)}
\PY{n}{problem}\PY{o}{.}\PY{n}{addVariable}\PY{p}{(}\PY{l+s+s2}{\PYZdq{}}\PY{l+s+s2}{x4}\PY{l+s+s2}{\PYZdq{}}\PY{p}{,} \PY{n+nb}{range}\PY{p}{(}\PY{l+m+mi}{67}\PY{p}{)}\PY{p}{)}

\PY{c+c1}{\PYZsh{} Creando las restricciones}
\PY{n}{problem}\PY{o}{.}\PY{n}{addConstraint}\PY{p}{(}\PY{k}{lambda} \PY{n}{x1}\PY{p}{,} \PY{n}{x2}\PY{p}{,} \PY{n}{x3}\PY{p}{,} \PY{n}{x4}\PY{p}{:} \PY{l+m+mi}{8}\PY{o}{*}\PY{n}{x1} \PY{o}{+} \PY{l+m+mf}{6.8}\PY{o}{*}\PY{n}{x2} \PY{o}{+} \PY{l+m+mi}{4}\PY{o}{*}\PY{n}{x3} \PY{o}{+} \PY{l+m+mi}{3}\PY{o}{*}\PY{n}{x4} \PY{o}{\PYZlt{}}\PY{o}{=} \PY{l+m+mi}{200}\PY{p}{,} \PY{p}{(}\PY{l+s+s2}{\PYZdq{}}\PY{l+s+s2}{x1}\PY{l+s+s2}{\PYZdq{}}\PY{p}{,} \PY{l+s+s2}{\PYZdq{}}\PY{l+s+s2}{x2}\PY{l+s+s2}{\PYZdq{}}\PY{p}{,} \PY{l+s+s2}{\PYZdq{}}\PY{l+s+s2}{x3}\PY{l+s+s2}{\PYZdq{}}\PY{p}{,} \PY{l+s+s2}{\PYZdq{}}\PY{l+s+s2}{x4}\PY{l+s+s2}{\PYZdq{}}\PY{p}{)}\PY{p}{)}
\PY{n}{problem}\PY{o}{.}\PY{n}{addConstraint}\PY{p}{(}\PY{k}{lambda} \PY{n}{x1}\PY{p}{,} \PY{n}{x2}\PY{p}{,} \PY{n}{x3}\PY{p}{,} \PY{n}{x4}\PY{p}{:} \PY{l+m+mi}{100}\PY{o}{*}\PY{n}{x1} \PY{o}{+} \PY{l+m+mi}{45}\PY{o}{*}\PY{n}{x2} \PY{o}{+} \PY{l+m+mi}{10}\PY{o}{*}\PY{n}{x3} \PY{o}{+} \PY{l+m+mi}{25}\PY{o}{*}\PY{n}{x4} \PY{o}{\PYZlt{}}\PY{o}{=} \PY{l+m+mi}{3000}\PY{p}{,} \PY{p}{(}\PY{l+s+s2}{\PYZdq{}}\PY{l+s+s2}{x1}\PY{l+s+s2}{\PYZdq{}}\PY{p}{,} \PY{l+s+s2}{\PYZdq{}}\PY{l+s+s2}{x2}\PY{l+s+s2}{\PYZdq{}}\PY{p}{,} \PY{l+s+s2}{\PYZdq{}}\PY{l+s+s2}{x3}\PY{l+s+s2}{\PYZdq{}}\PY{p}{,} \PY{l+s+s2}{\PYZdq{}}\PY{l+s+s2}{x4}\PY{l+s+s2}{\PYZdq{}}\PY{p}{)}\PY{p}{)}
\PY{n}{problem}\PY{o}{.}\PY{n}{addConstraint}\PY{p}{(}\PY{k}{lambda} \PY{n}{x1}\PY{p}{,} \PY{n}{x2}\PY{p}{,} \PY{n}{x3}\PY{p}{,} \PY{n}{x4}\PY{p}{:} \PY{l+m+mi}{10}\PY{o}{*}\PY{n}{x1} \PY{o}{+} \PY{l+m+mi}{7}\PY{o}{*}\PY{n}{x2} \PY{o}{+} \PY{l+m+mi}{3}\PY{o}{*}\PY{n}{x3} \PY{o}{+} \PY{l+m+mf}{4.5}\PY{o}{*}\PY{n}{x4} \PY{o}{\PYZlt{}}\PY{o}{=} \PY{l+m+mi}{1000}\PY{p}{,} \PY{p}{(}\PY{l+s+s2}{\PYZdq{}}\PY{l+s+s2}{x1}\PY{l+s+s2}{\PYZdq{}}\PY{p}{,} \PY{l+s+s2}{\PYZdq{}}\PY{l+s+s2}{x2}\PY{l+s+s2}{\PYZdq{}}\PY{p}{,} \PY{l+s+s2}{\PYZdq{}}\PY{l+s+s2}{x3}\PY{l+s+s2}{\PYZdq{}}\PY{p}{,} \PY{l+s+s2}{\PYZdq{}}\PY{l+s+s2}{x4}\PY{l+s+s2}{\PYZdq{}}\PY{p}{)}\PY{p}{)}

\PY{c+c1}{\PYZsh{} Obteniendo las soluciones factibles}
\PY{n}{solutions} \PY{o}{=} \PY{n}{problem}\PY{o}{.}\PY{n}{getSolutions}\PY{p}{(}\PY{p}{)}

\PY{c+c1}{\PYZsh{} Creando la función objetivo}
\PY{k}{def} \PY{n+nf}{fncn\PYZus{}obj}\PY{p}{(}\PY{n}{x1}\PY{p}{,} \PY{n}{x2}\PY{p}{,} \PY{n}{x3}\PY{p}{,} \PY{n}{x4}\PY{p}{)}\PY{p}{:}
    \PY{k}{return} \PY{l+m+mi}{20}\PY{o}{*}\PY{n}{x1} \PY{o}{+} \PY{l+m+mi}{16}\PY{o}{*}\PY{n}{x2} \PY{o}{+} \PY{l+m+mi}{9}\PY{o}{*}\PY{n}{x3} \PY{o}{+} \PY{l+m+mi}{7}\PY{o}{*}\PY{n}{x4}

\PY{n}{auxProblem} \PY{o}{=} \PY{n}{problem}

\PY{n+nb}{len}\PY{p}{(}\PY{n}{solutions}\PY{p}{)}
\end{Verbatim}
\end{tcolorbox}

            \begin{tcolorbox}[breakable, size=fbox, boxrule=.5pt, pad at break*=1mm, opacityfill=0]
\prompt{Out}{outcolor}{21}{\boxspacing}
\begin{Verbatim}[commandchars=\\\{\}]
126423
\end{Verbatim}
\end{tcolorbox}
        
    Existen un total de 126,423 soluciones factibles para este modelo

    \hypertarget{acotamiento-de-la-funciuxf3n-objetivo-para-encontrar-la-soluciuxf3n-uxf3ptima}{%
\subsection{Acotamiento de la función objetivo para encontrar la
solución
óptima}\label{acotamiento-de-la-funciuxf3n-objetivo-para-encontrar-la-soluciuxf3n-uxf3ptima}}

    De acuerdo con Frederick Hillier, la especialidad de la programación
restringida o la programación de restricciones son los problemas con
muchas restricciones y sin una función objetivo. Sin embargo, también
concede que la presencia de una función objetivo no es impedimento y, en
muchos casos, ayuda para restringir el conjunto factible del modelo aún
más (Hillier, 478).

Antes vimos que las variables \(x_1, x_2, x_3\) y \(x_4\) son tales que
\begin{align}
    x_1 &\in \{0, 1, 2, \dots, 25\}\\
    x_2 &\in \{0, 1, 2, \dots, 29\}\\
    x_3 &\in \{0, 1, 2, \dots, 50\}\\
    x_4 &\in \{0, 1, 2, \dots, 66\}
\end{align}
En particular, notemos que \(\bar{x}_1 = (0, 0, 50, 0)\) y
\(\bar{x}_2 = (0, 29, 0, 0)\) son soluciones factibles y que
\begin{equation}
z_1 = f(0, 0, 50, 0) = 450 < 464 = f(0, 29, 0, 0) = z_2
\end{equation}
lo cual es un enorme avance porque no solo podemos aseverar que el valor
óptimo de \(z\) es positivo, sino que debe satisfacer la restricción
\begin{equation}
    z \geq 450
\end{equation}
la cual nos permite descartar más de 100,000 soluciones factibles
adicionales, tal y como mostramos a continuación:

    \begin{tcolorbox}[breakable, size=fbox, boxrule=1pt, pad at break*=1mm,colback=cellbackground, colframe=cellborder]
\prompt{In}{incolor}{22}{\boxspacing}
\begin{Verbatim}[commandchars=\\\{\}]
\PY{n}{auxProblem}\PY{o}{.}\PY{n}{addConstraint}\PY{p}{(}\PY{k}{lambda} \PY{n}{x1}\PY{p}{,} \PY{n}{x2}\PY{p}{,} \PY{n}{x3}\PY{p}{,} \PY{n}{x4}\PY{p}{:} \PY{l+m+mi}{20}\PY{o}{*}\PY{n}{x1} \PY{o}{+} \PY{l+m+mi}{16}\PY{o}{*}\PY{n}{x2} \PY{o}{+} \PY{l+m+mi}{9}\PY{o}{*}\PY{n}{x3} \PY{o}{+} \PY{l+m+mi}{7}\PY{o}{*}\PY{n}{x4} \PY{o}{\PYZgt{}}\PY{o}{=} \PY{l+m+mi}{450}\PY{p}{,} \PY{p}{(}\PY{l+s+s2}{\PYZdq{}}\PY{l+s+s2}{x1}\PY{l+s+s2}{\PYZdq{}}\PY{p}{,} \PY{l+s+s2}{\PYZdq{}}\PY{l+s+s2}{x2}\PY{l+s+s2}{\PYZdq{}}\PY{p}{,} \PY{l+s+s2}{\PYZdq{}}\PY{l+s+s2}{x3}\PY{l+s+s2}{\PYZdq{}}\PY{p}{,} \PY{l+s+s2}{\PYZdq{}}\PY{l+s+s2}{x4}\PY{l+s+s2}{\PYZdq{}}\PY{p}{)}\PY{p}{)}
\PY{n}{solutions} \PY{o}{=} \PY{n}{auxProblem}\PY{o}{.}\PY{n}{getSolutions}\PY{p}{(}\PY{p}{)}
\PY{n+nb}{len}\PY{p}{(}\PY{n}{solutions}\PY{p}{)}
\end{Verbatim}
\end{tcolorbox}

            \begin{tcolorbox}[breakable, size=fbox, boxrule=.5pt, pad at break*=1mm, opacityfill=0]
\prompt{Out}{outcolor}{22}{\boxspacing}
\begin{Verbatim}[commandchars=\\\{\}]
20949
\end{Verbatim}
\end{tcolorbox}
        
    En realidad no hay razón para haberse conformado con probar un valor al
azar de alguna componente de \(\bar{x}\). Podemos ver que el mayor
coeficiente en la función objetivo corresponde a \(x_1\), por lo que
definimos la restricción
\begin{equation}
    z \geq f(25, 0 , 0, 0) = 500
\end{equation}
y tenemos que el conjunto factibles se ha reducido tanto que únicamente
nos queda una única solución

    \begin{tcolorbox}[breakable, size=fbox, boxrule=1pt, pad at break*=1mm,colback=cellbackground, colframe=cellborder]
\prompt{In}{incolor}{18}{\boxspacing}
\begin{Verbatim}[commandchars=\\\{\}]
\PY{n}{problem}\PY{o}{.}\PY{n}{addConstraint}\PY{p}{(}\PY{k}{lambda} \PY{n}{x1}\PY{p}{,} \PY{n}{x2}\PY{p}{,} \PY{n}{x3}\PY{p}{,} \PY{n}{x4}\PY{p}{:} \PY{l+m+mi}{20}\PY{o}{*}\PY{n}{x1} \PY{o}{+} \PY{l+m+mi}{16}\PY{o}{*}\PY{n}{x2} \PY{o}{+} \PY{l+m+mi}{9}\PY{o}{*}\PY{n}{x3} \PY{o}{+} \PY{l+m+mi}{7}\PY{o}{*}\PY{n}{x4} \PY{o}{\PYZgt{}}\PY{o}{=} \PY{l+m+mi}{500}\PY{p}{,} \PY{p}{(}\PY{l+s+s2}{\PYZdq{}}\PY{l+s+s2}{x1}\PY{l+s+s2}{\PYZdq{}}\PY{p}{,} \PY{l+s+s2}{\PYZdq{}}\PY{l+s+s2}{x2}\PY{l+s+s2}{\PYZdq{}}\PY{p}{,} \PY{l+s+s2}{\PYZdq{}}\PY{l+s+s2}{x3}\PY{l+s+s2}{\PYZdq{}}\PY{p}{,} \PY{l+s+s2}{\PYZdq{}}\PY{l+s+s2}{x4}\PY{l+s+s2}{\PYZdq{}}\PY{p}{)}\PY{p}{)}
\PY{n}{solutions} \PY{o}{=} \PY{n}{problem}\PY{o}{.}\PY{n}{getSolutions}\PY{p}{(}\PY{p}{)}
\PY{n+nb}{len}\PY{p}{(}\PY{n}{solutions}\PY{p}{)}
\end{Verbatim}
\end{tcolorbox}

            \begin{tcolorbox}[breakable, size=fbox, boxrule=.5pt, pad at break*=1mm, opacityfill=0]
\prompt{Out}{outcolor}{18}{\boxspacing}
\begin{Verbatim}[commandchars=\\\{\}]
1
\end{Verbatim}
\end{tcolorbox}
        
    y que, por supuesto, podemos corroborar es la solución óptima del modelo
que planteamos originalmente:

    \begin{tcolorbox}[breakable, size=fbox, boxrule=1pt, pad at break*=1mm,colback=cellbackground, colframe=cellborder]
\prompt{In}{incolor}{19}{\boxspacing}
\begin{Verbatim}[commandchars=\\\{\}]
\PY{n+nb}{print}\PY{p}{(}\PY{l+s+s2}{\PYZdq{}}\PY{l+s+s2}{El valor máximo es: }\PY{l+s+si}{\PYZob{}\PYZcb{}}\PY{l+s+s2}{ y la solución es }\PY{l+s+si}{\PYZob{}\PYZcb{}}\PY{l+s+s2}{\PYZdq{}}\PY{o}{.}\PY{n}{format}\PY{p}{(}\PY{n}{z}\PY{p}{,}\PY{p}{(}\PY{n}{solucion}\PY{p}{[}\PY{l+s+s1}{\PYZsq{}}\PY{l+s+s1}{x1}\PY{l+s+s1}{\PYZsq{}}\PY{p}{]}\PY{p}{,}\PY{n}{solucion}\PY{p}{[}\PY{l+s+s1}{\PYZsq{}}\PY{l+s+s1}{x2}\PY{l+s+s1}{\PYZsq{}}\PY{p}{]}\PY{p}{,}\PY{n}{solucion}\PY{p}{[}\PY{l+s+s1}{\PYZsq{}}\PY{l+s+s1}{x3}\PY{l+s+s1}{\PYZsq{}}\PY{p}{]}\PY{p}{,}\PY{n}{solucion}\PY{p}{[}\PY{l+s+s1}{\PYZsq{}}\PY{l+s+s1}{x4}\PY{l+s+s1}{\PYZsq{}}\PY{p}{]}\PY{p}{)}\PY{p}{)}\PY{p}{)}
\end{Verbatim}
\end{tcolorbox}

    \begin{Verbatim}[commandchars=\\\{\}]
El valor máximo es: 500 y la solución es (25, 0, 0, 0)
    \end{Verbatim}

    \begin{tcolorbox}[breakable, size=fbox, boxrule=1pt, pad at break*=1mm,colback=cellbackground, colframe=cellborder]
\prompt{In}{incolor}{28}{\boxspacing}
\begin{Verbatim}[commandchars=\\\{\}]
\PY{c+c1}{\PYZsh{} Creando el problema y el rango de las variables:}
\PY{n}{original} \PY{o}{=} \PY{n}{Problem}\PY{p}{(}\PY{p}{)}
\PY{n}{original}\PY{o}{.}\PY{n}{addVariables}\PY{p}{(}\PY{p}{[}\PY{l+s+s2}{\PYZdq{}}\PY{l+s+s2}{x1}\PY{l+s+s2}{\PYZdq{}}\PY{p}{,} \PY{l+s+s2}{\PYZdq{}}\PY{l+s+s2}{x2}\PY{l+s+s2}{\PYZdq{}}\PY{p}{,} \PY{l+s+s2}{\PYZdq{}}\PY{l+s+s2}{x3}\PY{l+s+s2}{\PYZdq{}}\PY{p}{,} \PY{l+s+s2}{\PYZdq{}}\PY{l+s+s2}{x4}\PY{l+s+s2}{\PYZdq{}}\PY{p}{]}\PY{p}{,} \PY{p}{[}\PY{n}{x1} \PY{o}{\PYZgt{}}\PY{o}{=} \PY{l+m+mi}{0}\PY{p}{,} \PY{n}{x2} \PY{o}{\PYZgt{}}\PY{o}{=} \PY{l+m+mi}{0}\PY{p}{,} \PY{n}{x3} \PY{o}{\PYZgt{}}\PY{o}{=} \PY{l+m+mi}{0}\PY{p}{,} \PY{n}{x4} \PY{o}{\PYZgt{}}\PY{o}{=} \PY{l+m+mi}{0}\PY{p}{]}\PY{p}{)}
\PY{c+c1}{\PYZsh{}problem.addVariables() }
\PY{c+c1}{\PYZsh{} Creando las restricciones}
\PY{n}{problem}\PY{o}{.}\PY{n}{addConstraint}\PY{p}{(}\PY{k}{lambda} \PY{n}{x1}\PY{p}{,} \PY{n}{x2}\PY{p}{,} \PY{n}{x3}\PY{p}{,} \PY{n}{x4}\PY{p}{:} \PY{l+m+mi}{8}\PY{o}{*}\PY{n}{x1} \PY{o}{+} \PY{l+m+mf}{6.8}\PY{o}{*}\PY{n}{x2} \PY{o}{+} \PY{l+m+mi}{4}\PY{o}{*}\PY{n}{x3} \PY{o}{+} \PY{l+m+mi}{3}\PY{o}{*}\PY{n}{x4} \PY{o}{\PYZlt{}}\PY{o}{=} \PY{l+m+mi}{200}\PY{p}{,} \PY{p}{(}\PY{l+s+s2}{\PYZdq{}}\PY{l+s+s2}{x1}\PY{l+s+s2}{\PYZdq{}}\PY{p}{,} \PY{l+s+s2}{\PYZdq{}}\PY{l+s+s2}{x2}\PY{l+s+s2}{\PYZdq{}}\PY{p}{,} \PY{l+s+s2}{\PYZdq{}}\PY{l+s+s2}{x3}\PY{l+s+s2}{\PYZdq{}}\PY{p}{,} \PY{l+s+s2}{\PYZdq{}}\PY{l+s+s2}{x4}\PY{l+s+s2}{\PYZdq{}}\PY{p}{)}\PY{p}{)}
\PY{n}{problem}\PY{o}{.}\PY{n}{addConstraint}\PY{p}{(}\PY{k}{lambda} \PY{n}{x1}\PY{p}{,} \PY{n}{x2}\PY{p}{,} \PY{n}{x3}\PY{p}{,} \PY{n}{x4}\PY{p}{:} \PY{l+m+mi}{100}\PY{o}{*}\PY{n}{x1} \PY{o}{+} \PY{l+m+mi}{45}\PY{o}{*}\PY{n}{x2} \PY{o}{+} \PY{l+m+mi}{10}\PY{o}{*}\PY{n}{x3} \PY{o}{+} \PY{l+m+mi}{25}\PY{o}{*}\PY{n}{x4} \PY{o}{\PYZlt{}}\PY{o}{=} \PY{l+m+mi}{3000}\PY{p}{,} \PY{p}{(}\PY{l+s+s2}{\PYZdq{}}\PY{l+s+s2}{x1}\PY{l+s+s2}{\PYZdq{}}\PY{p}{,} \PY{l+s+s2}{\PYZdq{}}\PY{l+s+s2}{x2}\PY{l+s+s2}{\PYZdq{}}\PY{p}{,} \PY{l+s+s2}{\PYZdq{}}\PY{l+s+s2}{x3}\PY{l+s+s2}{\PYZdq{}}\PY{p}{,} \PY{l+s+s2}{\PYZdq{}}\PY{l+s+s2}{x4}\PY{l+s+s2}{\PYZdq{}}\PY{p}{)}\PY{p}{)}
\PY{n}{problem}\PY{o}{.}\PY{n}{addConstraint}\PY{p}{(}\PY{k}{lambda} \PY{n}{x1}\PY{p}{,} \PY{n}{x2}\PY{p}{,} \PY{n}{x3}\PY{p}{,} \PY{n}{x4}\PY{p}{:} \PY{l+m+mi}{10}\PY{o}{*}\PY{n}{x1} \PY{o}{+} \PY{l+m+mi}{7}\PY{o}{*}\PY{n}{x2} \PY{o}{+} \PY{l+m+mi}{3}\PY{o}{*}\PY{n}{x3} \PY{o}{+} \PY{l+m+mf}{4.5}\PY{o}{*}\PY{n}{x4} \PY{o}{\PYZlt{}}\PY{o}{=} \PY{l+m+mi}{1000}\PY{p}{,} \PY{p}{(}\PY{l+s+s2}{\PYZdq{}}\PY{l+s+s2}{x1}\PY{l+s+s2}{\PYZdq{}}\PY{p}{,} \PY{l+s+s2}{\PYZdq{}}\PY{l+s+s2}{x2}\PY{l+s+s2}{\PYZdq{}}\PY{p}{,} \PY{l+s+s2}{\PYZdq{}}\PY{l+s+s2}{x3}\PY{l+s+s2}{\PYZdq{}}\PY{p}{,} \PY{l+s+s2}{\PYZdq{}}\PY{l+s+s2}{x4}\PY{l+s+s2}{\PYZdq{}}\PY{p}{)}\PY{p}{)}
\PY{n}{problem}\PY{o}{.}\PY{n}{addConstraint}\PY{p}{(}\PY{k}{lambda} \PY{n}{x1}\PY{p}{,} \PY{n}{x2}\PY{p}{,} \PY{n}{x3}\PY{p}{,} \PY{n}{x4}\PY{p}{:} \PY{l+m+mi}{20}\PY{o}{*}\PY{n}{x1} \PY{o}{+} \PY{l+m+mi}{16}\PY{o}{*}\PY{n}{x2} \PY{o}{+} \PY{l+m+mi}{9}\PY{o}{*}\PY{n}{x3} \PY{o}{+} \PY{l+m+mi}{7}\PY{o}{*}\PY{n}{x4} \PY{o}{\PYZgt{}}\PY{o}{=} \PY{l+m+mi}{0}\PY{p}{,} \PY{p}{(}\PY{l+s+s2}{\PYZdq{}}\PY{l+s+s2}{x1}\PY{l+s+s2}{\PYZdq{}}\PY{p}{,} \PY{l+s+s2}{\PYZdq{}}\PY{l+s+s2}{x2}\PY{l+s+s2}{\PYZdq{}}\PY{p}{,} \PY{l+s+s2}{\PYZdq{}}\PY{l+s+s2}{x3}\PY{l+s+s2}{\PYZdq{}}\PY{p}{,} \PY{l+s+s2}{\PYZdq{}}\PY{l+s+s2}{x4}\PY{l+s+s2}{\PYZdq{}}\PY{p}{)}\PY{p}{)}

\PY{c+c1}{\PYZsh{} Obteniendo las soluciones factibles}
\PY{n}{solutions} \PY{o}{=} \PY{n}{original}\PY{o}{.}\PY{n}{getSolution}\PY{p}{(}\PY{p}{)}

\PY{c+c1}{\PYZsh{} Creando la función objetivo}
\PY{k}{def} \PY{n+nf}{fncn\PYZus{}obj}\PY{p}{(}\PY{n}{x1}\PY{p}{,} \PY{n}{x2}\PY{p}{,} \PY{n}{x3}\PY{p}{,} \PY{n}{x4}\PY{p}{)}\PY{p}{:}
    \PY{k}{return} \PY{l+m+mi}{20}\PY{o}{*}\PY{n}{x1} \PY{o}{+} \PY{l+m+mi}{16}\PY{o}{*}\PY{n}{x2} \PY{o}{+} \PY{l+m+mi}{9}\PY{o}{*}\PY{n}{x3} \PY{o}{+} \PY{l+m+mi}{7}\PY{o}{*}\PY{n}{x4}

\PY{n+nb}{print}\PY{p}{(}\PY{l+s+s2}{\PYZdq{}}\PY{l+s+s2}{El valor máximo del modelo de programación lineal entera es: }\PY{l+s+si}{\PYZob{}\PYZcb{}}\PY{l+s+s2}{ y la solución es }\PY{l+s+si}{\PYZob{}\PYZcb{}}\PY{l+s+s2}{\PYZdq{}}\PY{o}{.}\PY{n}{format}\PY{p}{(}\PY{n}{z}\PY{p}{,}\PY{p}{(}\PY{n}{solucion}\PY{p}{[}\PY{l+s+s1}{\PYZsq{}}\PY{l+s+s1}{x1}\PY{l+s+s1}{\PYZsq{}}\PY{p}{]}\PY{p}{,}\PY{n}{solucion}\PY{p}{[}\PY{l+s+s1}{\PYZsq{}}\PY{l+s+s1}{x2}\PY{l+s+s1}{\PYZsq{}}\PY{p}{]}\PY{p}{,}\PY{n}{solucion}\PY{p}{[}\PY{l+s+s1}{\PYZsq{}}\PY{l+s+s1}{x3}\PY{l+s+s1}{\PYZsq{}}\PY{p}{]}\PY{p}{,}\PY{n}{solucion}\PY{p}{[}\PY{l+s+s1}{\PYZsq{}}\PY{l+s+s1}{x4}\PY{l+s+s1}{\PYZsq{}}\PY{p}{]}\PY{p}{)}\PY{p}{)}\PY{p}{)}
\end{Verbatim}
\end{tcolorbox}

    \begin{Verbatim}[commandchars=\\\{\}]
El valor máximo del modelo de programación lineal entera es: 500 y la solución
es (25, 0, 0, 0)
    \end{Verbatim}

    \hypertarget{bibliografuxeda}{%
\section{Bibliografía}\label{bibliografuxeda}}

    \begin{enumerate}
\def\labelenumi{\arabic{enumi}.}
\tightlist
\item
  Hillier, Frederick S., and Gerald J. Lieberman. \emph{Introducción a
  La investigación De Operaciones}. 9th ed. México D.F.: McGraw-Hill,
  2010.
\end{enumerate}


    % Add a bibliography block to the postdoc
    
    
    
\end{document}
